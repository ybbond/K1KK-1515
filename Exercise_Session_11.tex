\documentclass[12pt]{article}

\usepackage[paperwidth=21cm, paperheight=29.7cm, margin=1.5cm]{geometry}
\usepackage{xcolor}

\usepackage{graphicx}
\usepackage{wrapfig}

\usepackage{hyperref}
\hypersetup{colorlinks=true, urlcolor=blue}

\usepackage{amsmath}

\newcommand{\threestars}{\begin{center}$ {\ast}\,{\ast}\,{\ast} $\end{center}}
\newcommand{\theanswer}[1]{\colorbox{yellow}{#1}}
\newcommand{\given}[1]{\colorbox{lime}{#1}}
\newcommand{\Deriv}[2]{\ensuremath{\partial#1/\partial#2}}
\newcommand{\deriv}[2]{\ensuremath{\frac{\partial#1}{\partial#2}}}
\newcommand{\DerivByT}[1]{\ensuremath{\Deriv{#1}{t}}}
\newcommand{\derivByT}[1]{\ensuremath{\deriv{#1}{t}}}

\title{Exercises for Kalkulus 1 K1KK 1515 Session 10 \& 11}
\author{Yohanes Bandung Bondowoso (01085250015)}
\date{}

\begin{document}
\pagenumbering{gobble}
\maketitle


\noindent These exercises \underline{\textbf{originally}} is a requirement for attendance of asynchronous activity on Session 10.
\newline Then I continue for the task of synchronous activity on Session 11.
\newline I will solve all of the exercises for section \textbf{§Rates Equations}.


\vspace{2em}


\begin{wrapfigure}[9]{l}{0.18\textwidth}
  \includegraphics[width=0.18\textwidth]{./assets/e11_1.png}
\end{wrapfigure}
\noindent\textbf{Exercise 1.} The voltage $V$ (volts), current $I$ (amperes), and resistance $R$ (ohms)
of an electric circuit like the one shown here are related by the equation $V = IR$. Suppose that $V$ is increasing
at the rate of 1 volt/sec while $I$ is decreasing at the rate of 1/3 amp/sec. Let $t$ denote time in seconds.
\begin{enumerate}
  \item What is the value of \DerivByT{V}?
  \item What is the value of \DerivByT{I}?
  \item What equation relates \DerivByT{R} to \DerivByT{V} and \DerivByT{I}?
  \item Find the rate at which $R$ is changing when $V = 12 volts$ and $I = 2 amp$.
        Is $R$ increasing, or decreasing?
\end{enumerate}

\threestars

\noindent I will write differentiation with the \textbf{Lagrange's notation}. So for example \DerivByT{V} will be $V'$.
\newline Given the problem definition, we know the answer for (1) and (2).

\begin{enumerate}
  \item \theanswer{$V' = 1 V/s$}
    \newline As the value is \textit{increasing}, it is \textbf{positive}.
  \item \theanswer{$I' = -\frac{1}{3} A/s$}
    \newline As the value is \textit{decreasing}, it is \textbf{negative}.
  \item Given \textbf{Ohm's Law} $\fbox{$V=IR$}$, we differentiate with
    \textbf{Product Rule} $\fbox{$(uv)' = u'v + uv'$}$.
    $$V' = I'R + IR' \Rightarrow \theanswer{$R' = \frac{V' - I'R}{I}$}$$
  \item Given \given{$V = 12V$} and \given{$I = 2A$}, we can calculate
    $R = \frac{V}{I} = \frac{12}{2} \Rightarrow \given{$R = 6 \Omega$}$:
    $$
    R' = \frac{V' - I'R}{I} = \frac{1 - (-\frac{1}{3})6}{2}
    = \frac{1 - (-2)}{2} = \frac{3}{2} = \theanswer{$1,5 \Omega/s$}
    $$
    As the result is \textbf{positive}, it is \textit{increasing}.
\end{enumerate}

\noindent In this exercise, I implemented differentiation with \textbf{Product Rule}
for known formula \textbf{Ohm's Law}.
\newline I learnt that derivative with respect of time $t$ will result in the value of its changing rate.


\newpage


\begin{wrapfigure}{l}{0.25\textwidth}
  \includegraphics[width=0.25\textwidth]{./assets/e11_2.png}
\end{wrapfigure}
\noindent\textbf{Exercise 2.} A 13-ft ladder is leaning against a house when its base starts to slide away.
By the time the base is 12 ft from the house, the base is moving at the rate of 5 ft/sec.
\begin{enumerate}
  \item How fast is the top of the ladder sliding down the wall then?
  \item At what rate is the area of the triangle formed by the ladder, wall, and ground changing then?
  \item At what rate is the angle $\theta$ between the ladder and the ground changing then?
\end{enumerate}

\threestars

\noindent Given the problem definition, we know that \given{$s = 13 ft$}, \given{$x = 12 ft$}, and \given{$x' = 5 ft/s$}.
\newline We need to look for the current $y$. We know the \textbf{Triple Pythagoras} of $(5, 12, 13)$,
so $y$ must be $5$. But let's proof that:
$$s^2 = x^2 + y^2 \Rightarrow y^2 = s^2 - x^2 \Rightarrow y = \sqrt{s^2 - x^2}$$
$$y = \sqrt{13^2 - 12^2} = \sqrt{169 - 144} = \sqrt{25} \Rightarrow \given{$y = 5 ft$}$$

\begin{enumerate}
  \item Given the equation for \textbf{Pythagorean Theorem} $\fbox{$s^2 = x^2 + y^2$}$.
    \newline We know that the value of $s$ is constant (as it is an object, a ladder) so it
    will not be differentiated with respect of time $t$.

    $$13^2 = x^2 + y^2 \Rightarrow 0 = 2xx' + 2yy'$$
    $$0 = xx' + yy' \Rightarrow yy' = -xx' \Rightarrow y' = - \frac{xx'}{y}$$
    $$y' = - \frac{12 \times 5}{5} \Rightarrow \theanswer{$y' = -12 ft/sec$}$$

    As the value is \textbf{negative}, it is \textit{decreasing} or going down.
  \item Given the equation of triangle area is half of height multiplied by base.
    This is a right-angled triangle (has one 90\textdegree corner), so the formula based on known value is:
    $\fbox{$A = \frac{1}{2} x y$}$

    $$A = \frac{1}{2} x y \Rightarrow A' = \frac{1}{2} (xy' + x'y)$$
    $$
    A' = \frac{1}{2} (12 \times -12 + 5 \times 5) = \frac{1}{2} (-144 + 25) = \frac{1}{2} (-119)
    \Rightarrow \theanswer{$A' = -59,5ft^2/sec$}
    $$

    As the value is \textbf{negative}, it is \textit{decreasing}.
  \item Given we know that $s$ is constant,
    we can use either \textcircled{1} $\fbox{$cos\theta = \frac{x}{s}$}$
    or \textcircled{2} $\fbox{$sin\theta = \frac{y}{s}$}$ formula.

    To start, we use derivative from \textcircled{1} using
    \textbf{Quotient Rule} $\fbox{$(\frac{u}{v})' = \frac{u'v - uv'}{v^2}$}$:
    $$
    cos\theta = \frac{x}{s} \Rightarrow cos\theta = \frac{x}{13}
    \Rightarrow -sin\theta\theta' = \frac{x' \times 13 - x \times 0}{13^2}
    \Rightarrow -sin\theta\theta' = \frac{x' \times 13}{13 \times 13}
    $$
    In the next step, notice that we substitute $-sin\theta$ with \textcircled{2} to be $-\frac{y}{s}$:
    $$
    \theta' = \frac{x'}{13 \times -sin\theta}
    = \frac{x'}{13 \times -\frac{y}{s}}
    = \frac{5}{13 \times -\frac{5}{13}}
    = -\frac{5}{5}
    \Rightarrow \theanswer{$\theta' = -1 rad/s$}
    $$
\end{enumerate}


\newpage


\begin{wrapfigure}[6]{l}{0.25\textwidth}
  \includegraphics[width=0.25\textwidth]{./assets/e11_3.png}
\end{wrapfigure}
\noindent\textbf{Exercise 3.} A dinghy is pulled toward a dock by a rope from the bow through a ring on the dock 6ft above the bow.
The rope is hauled in at the rate of 2 ft/sec.
\begin{enumerate}
  \item How fast is the boat approaching the dock when 10 ft of rope are out?
  \item At what rate is the angle $\theta$ changing then (see the figure)?
\end{enumerate}

\threestars

\noindent Given the problem definition, we know that \given{$y = 6 ft$}, \given{$s' = -2 ft/s$}, and \given{$s = 10 ft$}.
\newline We know that $y$ is a constant value.
\newline We need to look for the current $x$. We know the \textbf{Triple Pythagoras} of $(3, 4, 5)$,
and the ratio suffice. So $x$ must be $8$. But let's proof that:
$$s^2 = x^2 + y^2 \Rightarrow x^2 = s^2 - y^2 \Rightarrow x = \sqrt{s^2 - y^2}$$
$$x = \sqrt{10^2 - 6^2} = \sqrt{100 - 36} = \sqrt{64} \Rightarrow \given{$x = 8 ft$}$$

\begin{enumerate}
  \item We will use \textbf{Pythagorean Theorem} $\fbox{$s^2 = x^2 + y^2$}$.
    \newline We know earlier that $y$ is constant.

    $$s^2 = x^2 + 6^2 \Rightarrow 2ss' = 2xx' + 0 \Rightarrow ss' = xx' \Rightarrow x' = \frac{ss'}{x}$$
    $$x' = \frac{10 \times -2}{8} = -\frac{20}{8} \Rightarrow \theanswer{$x' = -2,5 ft/s$}$$

    As the value is \textbf{negative}, it is \textit{decreasing}.
  \item Given we know that $y$ is constant,
    we can use either \textcircled{1} $\fbox{$cos\theta = \frac{y}{s}$}$
    or \textcircled{2} $\fbox{$sin\theta = \frac{x}{s}$}$ formula.

    To start, we use derivative from \textcircled{1} using
    \textbf{Quotient Rule} $\fbox{$(\frac{u}{v})' = \frac{u'v - uv'}{v^2}$}$:
    $$
    cos\theta = \frac{y}{s}
    \Rightarrow cos\theta = \frac{6}{s}
    \Rightarrow -sin\theta\theta' = \frac{0 \times 10 - 6 \times (-2)}{10^2}
    \Rightarrow -sin\theta\theta' = \frac{12}{100}
    $$
    In the next step, notice that we substitute $-sin\theta$ with \textcircled{2} to be $-\frac{x}{s}$:
    $$
    \theta' = \frac{12}{100 \times (- sin\theta)}
    = \frac{12}{100 \times (- \frac{x}{s})}
    = \frac{12}{100 \times (- \frac{8}{10})}
    = - \frac{12}{80}
    \Rightarrow \theanswer{$\theta' = -0,15rad/s$}
    $$
\end{enumerate}


\newpage


\begin{wrapfigure}[6]{l}{0.2\textwidth}
  \includegraphics[width=0.2\textwidth]{./assets/e11_4.png}
\end{wrapfigure}
\noindent\textbf{Exercise 4.} A balloon is rising vertically above a level, straight road at a constant rate of 1 ft/sec.
\newline Just when the balloon is 65 ft above the ground, a bicycle moving at a constant rate of 17 ft/sec passes under it.
\newline How fast is the distance s(t) between the bicycle and balloon increasing 3 sec later?

\vspace{4em}
\threestars
\vspace{2em}

\noindent Given the problem, we know that we have to count after 3 seconds.
\newline Also we know that \given{$y_{before} = 65ft$}, \given{$y' = 1ft/sec$}, and \given{$x' = 17 ft/sec$}.
\newline We need to know the value of $x$ and $y$ after 3 seconds.

$$x = x' \times 3 = 17 \times 3 \Rightarrow \given{$x = 51 ft$}$$
$$y = (y' \times 3) + y_{before} = (1 \times 3) + 65 \Rightarrow \given{$y = 68 ft$}$$

\noindent Given $x$ and $y$ is known, we need to find the diagonal distance $s$ using \textbf{Pythagorean Theorem}.

$$s^2 = x^2 + y^2 \Rightarrow s = \sqrt{x^2 + y^2}$$
$$s = \sqrt{51^2 + 68^2} = \sqrt{2601 + 4624} = \sqrt{7225} \Rightarrow \given{$s = 85ft$}$$

\noindent The problem is to find the speed of diagonal distance change $s'$.
\newline We use the differentiation of \textbf{Pythagorean Theorem} with respect of time.

$$s^2 = x^2 + y^2 \Rightarrow 2ss' = 2xx' + 2yy' \Rightarrow ss' = xx' + yy' \Rightarrow s' = \frac{xx' + yy'}{s}$$
$$s' = \frac{51 \times 17 + 68 \times 1}{85} = \frac{935}{85} \Rightarrow \theanswer{$s' = 11 ft/sec$}$$


\newpage


\section*{Supplements}

\subsection*{Derivative of Pythagorean Theorem}

\begin{align}
  s^2 &= x^2 + y^2 \\
  \derivByT{} s^2 &= \derivByT{} (x^2 + y^2) \\
  \deriv{}{s}s^2\deriv{s}{t} &= \deriv{}{x}x^2\deriv{x}{t} + \deriv{}{y}y^2\deriv{y}{t} \\
  2s^{2-1}\deriv{s}{t} &= 2x^{2-1}\deriv{x}{t} + 2y^{2-1}\deriv{y}{t} \\
  2s\deriv{s}{t} &= 2x\deriv{x}{t} + 2y\deriv{y}{t} \\
  2ss' &= 2xx' + 2yy'
\end{align}

\begin{enumerate}
  \item Base \textbf{Pythagorean Theorem}.
  \item We implement derivative with respect of time \derivByT{} on both side.
  \item We expand the differentiation, also as it is implicit differentiation,
    we add respecting differentiation with same unit.
  \item We implement the differentiation, and keep the derived base unit.
  \item This is the main result of differentiation in \textbf{Leibniz's notation}\footnote{source: Wikipedia \href{https://en.wikipedia.org/wiki/Notation_for_differentiation\#Leibniz's_notation}{"Notation for differentiation" §Leibniz's notation}}.
  \item This is the \textbf{Lagrange's notation}\footnote{source: Wikipedia \href{https://en.wikipedia.org/wiki/Notation_for_differentiation\#Lagrange's_notation}{"Notation for differentiation" §Lagrange's notation}}.
\end{enumerate}


\end{document}
