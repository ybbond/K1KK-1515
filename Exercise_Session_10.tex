\documentclass[12pt]{article}

\usepackage[paperwidth=21cm, paperheight=29.7cm, margin=1.5cm]{geometry}
\usepackage{xcolor}

\usepackage{graphicx}
\usepackage{wrapfig}

\usepackage{amsmath}

\newcommand{\threestars}{\begin{center}$ {\ast}\,{\ast}\,{\ast} $\end{center}}
\newcommand{\theanswer}[1]{\colorbox{yellow}{#1}}
\newcommand{\given}[1]{\colorbox{lime}{#1}}
\newcommand{\Deriv}[2]{\ensuremath{\partial#1/\partial#2}}
\newcommand{\deriv}[2]{\ensuremath{\frac{\partial#1}{\partial#2}}}
\newcommand{\DerivByT}[1]{\ensuremath{\Deriv{#1}{t}}}
\newcommand{\derivByT}[1]{\ensuremath{\deriv{#1}{t}}}

\title{Exercise for Kalkulus 1 K1KK 1515 Session 10}
\author{Yohanes Bandung Bondowoso (01085250015)}
\date{}

\begin{document}
\pagenumbering{gobble}
\maketitle


\noindent This exercise is a requirement for attendance as part of asynchronous activity on Session 10.
I will solve the exercise for the first section of \textbf{Rates Equations}.


\vspace{2em}


\begin{wrapfigure}[9]{l}{0.18\textwidth}
  \includegraphics[width=0.18\textwidth]{./assets/e11_1.png}
\end{wrapfigure}
\noindent\textbf{Exercise 1.} The voltage $V$ (volts), current $I$ (amperes), and resistance $R$ (ohms)
of an electric circuit like the one shown here are related by the equation $V = IR$. Suppose that $V$ is increasing
at the rate of 1 volt/sec while $I$ is decreasing at the rate of 1/3 amp/sec. Let $t$ denote time in seconds.
\begin{enumerate}
  \item What is the value of \DerivByT{V}?
  \item What is the value of \DerivByT{I}?
  \item What equation relates \DerivByT{R} to \DerivByT{V} and \DerivByT{I}?
  \item Find the rate at which $R$ is changing when $V = 12 volts$ and $I = 2 amp$.
        Is $R$ increasing, or decreasing?
\end{enumerate}


\threestars


\noindent I will write differentiation with the \textbf{Lagrange's notation}. So for example \DerivByT{V} will be $V'$.
\newline Given the problem definition, we know the answer for (1) and (2).

\begin{enumerate}
  \item \theanswer{$V' = 1 V/s$}
    \newline As the value is \textit{increasing}, it is \textbf{positive}.
  \item \theanswer{$I' = -\frac{1}{3} A/s$}
    \newline As the value is \textit{decreasing}, it is \textbf{negative}.
  \item Given \textbf{Ohm's Law} $\fbox{$V=IR$}$, we differentiate with
    \textbf{Product Rule} $\fbox{$(uv)' = u'v + uv'$}$.
    $$V' = I'R + IR' \Rightarrow \theanswer{$R' = \frac{V' - I'R}{I}$}$$
  \item Given \given{$V = 12V$} and \given{$I = 2A$}, we can calculate
    $R = \frac{V}{I} = \frac{12}{2} \Rightarrow \given{$R = 6 \Omega$}$:
    $$
    R' = \frac{V' - I'R}{I} = \frac{1 - (-\frac{1}{3})6}{2}
    = \frac{1 - (-2)}{2} = \frac{3}{2} = \theanswer{$1,5 \Omega/s$}
    $$
    As the result is \textbf{positive}, it is \textit{increasing}.
\end{enumerate}

\noindent In this exercise, I implemented differentiation with \textbf{Product Rule}
for known formula \textbf{Ohm's Law}.
\newline I learnt that derivative with respect of time $t$ will result in the value of its changing rate.



\end{document}
